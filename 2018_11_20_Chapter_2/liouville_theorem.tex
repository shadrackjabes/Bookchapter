\documentclass[a4paper]{article}
\usepackage{amsmath}
\usepackage{amssymb}
\usepackage{mathrsfs}
\usepackage[margin=0.5in]{geometry}
\title{Liouville Theorem}
\date{}
\begin{document}
\maketitle
\section{Equations of motion}
In an equilibrium system, the path traced out between two state points is determined by the extrema of the function known as action, $S$ which is defined by the following expression
\begin{equation}
S=\int^{t_{1}}_{t_{2}}\mathcal{L}(\dot{q}_{i},q_{i},t)~{\rm d}t,
\end{equation}
where $q_{i}=\{q_{1}\cdot\cdot\cdot q_{n}\},\dot{q}_{i}=\{\dot{q}_{1}\cdot\cdot\cdot \dot{q}_{n}\}$, $\mathcal{L}$ is the Lagrangian of the system and is given by
\begin{equation}\label{eq:lagrangian}
\begin{split}
\mathcal{L}(\dot{q}_{i},q_{i},t)&=KE-PE=\sum_{i}\frac{m\dot{q}_{i}^{2}}{2} - V(q_{1}...q_{n}),\\
\frac{\partial \mathcal{L}}{\partial \dot{q}_{i}}&=m\dot{q}_{\rm i}=p_{i},~~\frac{\partial \mathcal{L}}{\partial q_{i}}=-\frac{\partial V(q_{1}...q_{n})}{\partial q_{i}}=\dot{p_{i}}.
\end{split}
\end{equation}
Let us consider a system which moves from a position $q_{i}$ at $t_{1}$ to $q_{i}^{\prime}$ at $t_{2}$. There can be several paths which can be traced out by the system. The true path corresponds to the one for which the action, $S$ is an extrema. This can be calculated in the following way,
\begin{equation}
\begin{split}
\delta S&=\int^{t_{1}}_{t_{2}}\delta\mathcal{L}(\dot{q}_{i},q_{i},t)~{\rm d}t=\int^{t_{1}}_{t_{2}}\left(\frac{\partial\mathcal{L}(\dot{q}_{i},q_{i},t)~}{\partial q_{\rm i}}\delta q_{\rm i}+\frac{\partial\mathcal{L}(\dot{q}_{i},q_{i},t)~}{\partial \dot{q}_{\rm i}}\delta \dot{q}\right){\rm d}t=0,
\end{split}
\end{equation}
where $\delta q$ and $\delta \dot{q}$ are time dependent functions which represent the position and velocity differences between any two different paths, respectively.  The above expression can be further simplified in the following way,
\begin{equation}\label{eq:leastaction}
\begin{split}
\delta S&=\int^{t_{1}}_{t_{2}}\left(\frac{\partial\mathcal{L}(\dot{q}_{i},q_{i},t)~}{\partial q_{\rm i}}\delta q_{\rm i}+\frac{\partial\mathcal{L}(\dot{q}_{i},q_{i},t)~}{\partial \dot{q}_{\rm i}}\delta \dot{q}\right){\rm d}t,\\
&=\int^{t_{1}}_{t_{2}}\left(\frac{\partial\mathcal{L}(\dot{q}_{i},q_{i},t)~}{\partial q_{\rm i}}\delta q_{\rm i}dt\right) +\left(\frac{\partial\mathcal{L}(\dot{q}_{i},q_{i},t)~}{\partial {q}_{\rm i}}\delta {q}\right)\Bigg|^{t_{1}}_{t_{2}}{\rm d}t-\int^{t_{1}}_{t_{2}}\left(\frac{\rm d}{{\rm d}t}\frac{\partial\mathcal{L}(\dot{q}_{i},q_{i},t)~}{\partial \dot{q}_{\rm i}}\delta q_{\rm i}dt\right),\\
&=\int^{t_{1}}_{t_{2}}\left(\frac{\partial\mathcal{L}(\dot{q}_{i},q_{i},t)~}{\partial q_{\rm i}} - \frac{\rm d}{{\rm d}t}\frac{\partial\mathcal{L}(\dot{q}_{i},q_{i},t)~}{\partial \dot{q}_{\rm i}}\right)\delta q_{\rm i}dt=0,\\
\end{split}
\end{equation}
where $\delta q(t_{1})=\delta q(t_{2})=0$ as all paths have the same initial and final points. The above equality is valid for all values of $\delta q_{i}$, therefore the term within the brackets is equal to zero,
\begin{equation}
\frac{\rm d}{{\rm d}t}\frac{\partial\mathcal{L}(\dot{q}_{i},q_{i},t)~}{\partial \dot{q}_{\rm i}}=\frac{\partial\mathcal{L}(\dot{q}_{i},q_{i},t)~}{\partial q_{\rm i}}.
\end{equation}
To understand this in a better way, let us consider the case of a free particle ($PE=0$). Combining the above equation with Eq.~(\ref{eq:lagrangian}), it can be seen that,
\begin{equation}
\begin{split}
\frac{\rm d}{{\rm d}t}\frac{\partial\mathcal{L}(\dot{q}_{i},q_{i},t)~}{\partial \dot{q}_{\rm i}}&=m\frac{d\dot{q}}{dt},\\
\frac{\partial\mathcal{L}(\dot{q}_{i},q_{i},t)~}{\partial q_{\rm i}}&=0,\\
\frac{d\dot{q}}{dt}=0.
\end{split}
\end{equation}
The above framework shows that the true path is the one where the particle has a uniform velocity which corresponds to reality. For any system, the equations of motion can be derived using the following equations,
\begin{equation}\label{eq:equationsmotion}
\begin{split}
\frac{\partial \mathcal{L}}{\partial \dot{q}_{i}}&=m\dot{q}_{\rm i},\\\frac{\partial \mathcal{L}}{\partial q_{i}}&=-\frac{\partial V(q_{1}...q_{n})}{\partial q_{i}},\\
\frac{\rm d}{{\rm d}t}\frac{\partial\mathcal{L}(\dot{q}_{i},q_{i},t)~}{\partial \dot{q}_{\rm i}}&=\frac{\partial\mathcal{L}(\dot{q}_{i},q_{i},t)~}{\partial q_{\rm i}}.
\end{split}
\end{equation} 
\section{Canonical transformations}
Let us consider two sets of variables namely $q_{i},\dot{q}_{i}$ and $Q_{i},\dot{Q}_{i}$. Further, let the corresponding Lagrangians be $\mathcal{L}(q_{i},\dot{q}_{i},t)$ and $\mathcal{L}^{\prime}(Q_{i},\dot{Q}_{i},t)$. Then it can be shown that the equations of motions are same as long as the two Lagrangians can related by a total time derivative of a function. To prove this, let us consider the following relation between $\mathcal{L}$ and $\mathcal{L}^{\prime}$,
\begin{equation}\label{eq:canonical}
\mathcal{L}^{\prime}(Q_{i},\dot{Q}_{i},t)=\mathcal{L}(q_{i},\dot{q}_{i},t) + \frac{{\rm d}f(q_{i},Q_{i},t)}{{\rm d}t}.
\end{equation}
For the set $Q_{i},P_{i}$, the expression corresponding to the minimization of the action is
\begin{equation}
\begin{split}
S^{\prime}&=\int^{t_{1}}_{t_{2}}\mathcal{L}^{\prime}(Q_{i},\dot{Q}_{i},t)~{\rm d}t=\int^{t_{1}}_{t_{2}}\left(\mathcal{L}(q_{i},\dot{q}_{i},t) + \frac{{\rm d}f(q_{i},t)}{{\rm d}t}\right){\rm d}t,
\\
&=\int^{t_{1}}_{t_{2}}\mathcal{L}(q_{i},\dot{q}_{i},t){\rm d}t + f(q_{i},t)\Bigg|_{t_{1}}^{t_{2}}=\int^{t_{1}}_{t_{2}}\mathcal{L}(q_{i},\dot{q}_{i},t){\rm d}t + {\rm constant},\\
\delta S^{\prime}&=\int^{t_{1}}_{t_{2}}\delta\mathcal{L}(q_{i},\dot{q}_{i},t){\rm d}t=\delta S.
\end{split}
\end{equation}
The transformations which can be expressed in terms of the form given in Eq.~(\ref{eq:canonical}) are known as canonical transformations. The funtion $f$ is known as the generating function and has the following form,
\begin{equation}
{\rm d}f(q_{i},Q_{i},t)=(\mathcal{L}^{\prime}-\mathcal{L}){\rm d}t.
\end{equation}
Before looking at the properties of the generating function, let us legendre transform the Lagrangian to the Hamiltonian where the latter is a function of the positions and momenta,
\begin{equation}
\begin{split}
{\rm d}\mathcal{L}(q_{i},\dot{q}_{i})&=\frac{\partial \mathcal{L}}{\partial q_{i}} {\rm d}q_{i}+\frac{\partial \mathcal{L}}{\partial \dot{q}_{i}} {\rm d}\dot{q}_{i},\\
&=\dot{p}_{i} {\rm d}q_{i}+p_{i} {\rm d}\dot{q}_{i},\\
\mathcal{H}&=p_{i}\dot{q}_{i}-\mathcal{L}=\frac{p_{i}\dot{q}_{i}}{2} + V(q_{1}...q_{n})=KE+PE,\\
{\rm d}\mathcal{H}&=p_{i}{\rm d}\dot{q}_{i}+{\rm d}p_{i} \dot{q}_{i}-{\rm d}\mathcal{L},\\
&=\dot{q}_{i}{\rm d}p_{i}- \dot{p}_{i}{\rm d}q_{i},\\
\frac{\partial \mathcal{H}}{\partial q_{i}}&=-\dot{p}_{i}=\frac{\partial V(q_{1}...q_{n})}{\partial q_{i}},~~\frac{\partial \mathcal{H}}{\partial p_{i}}=\dot{q}_{i}.\\
\end{split}
\end{equation}
Now, the generating function $f$ has the following form,
\begin{equation}\label{eq:generatingfunction}
\begin{split}
{\rm d}f(q_{i},Q_{i},t)&=(P_{i}\dot{Q}_{i}-\mathcal{H}^{\prime}-p_{i}\dot{q}_{i}+\mathcal{H}){\rm d}t,\\
&=P_{i}{\rm d}Q_{i}-p_{i}{\rm d}q_{i}+(\mathcal{H}-\mathcal{H}^{\prime}){\rm d}t,\\
\frac{\partial f}{\partial q_{i}}&=-p_{i},~~\frac{\partial f}{\partial Q_{i}}=P_{i},~~\frac{\partial f}{\partial t}=\mathcal{H}-\mathcal{H}^{\prime}.
\end{split}
\end{equation}
A point to note here is that one can define three other generating functions in terms of other variables through Legendre transformation of the above function ($F_{1}$ from now on)
\begin{equation}\label{eq:generatingfunctions}
\begin{split}
{\rm d}F_{2}(q_{i}, P_{i},t)&={\rm d}\left(F_{1}-P_{i}Q_{i}\right)=-Q_{i}{\rm d}P_{i}-p_{i}{\rm d}q_{i}+(\mathcal{H}-\mathcal{H}^{\prime}){\rm d}t,\\
\frac{\partial F_{2}}{\partial q_{i}}&=-p_{i},~~\frac{\partial F_{2}}{\partial P_{i}}=-Q_{i},~~\frac{\partial F_{2}}{\partial t}=\mathcal{H}-\mathcal{H}^{\prime},\\
{\rm d}F_{3}(Q_{i}, p_{i},t)&={\rm d}\left(F_{1}+p_{i}q_{i}\right)=P_{i}{\rm d}Q_{i}+q_{i}{\rm d}p_{i}+(\mathcal{H}-\mathcal{H}^{\prime}){\rm d}t,\\
\frac{\partial F_{3}}{\partial Q_{i}}&=P_{i},~~\frac{\partial F_{3}}{\partial p_{i}}=q_{i},~~\frac{\partial F_{3}}{\partial t}=\mathcal{H}-\mathcal{H}^{\prime},\\
{\rm d}F_{4}(P_{i}, p_{i},t)&={\rm d}\left(F_{1}+p_{i}q_{i}-P_{i}Q_{i}\right)=-Q_{i}{\rm d}P_{i}+q_{i}{\rm d}p_{i}+(\mathcal{H}-\mathcal{H}^{\prime}){\rm d}t,\\
\frac{\partial F_{4}}{\partial P_{i}}&=-Q_{i},~~\frac{\partial F_{4}}{\partial p_{i}}=q_{i},~~\frac{\partial F_{4}}{\partial t}=\mathcal{H}-\mathcal{H}^{\prime}.
\end{split}
\end{equation} 
The generating functions $f_{1}$ and $f_{4}$ cannot perform identity transformations. Before moving forward, let us consider the effect of scaling a Lagrangian $\mathcal{L}$ by a constant such that $\mathcal{L}^{\prime}=\lambda\mathcal{L}$, it can be seen in a straight forward way from Eq.~(\ref{eq:leastaction}) that the equations of motion remain same. Combining this and Eq.~(\ref{eq:generatingfunction}), one can define a broader generating function in the following way,
\begin{equation}\label{eq:generatingfunction}
\begin{split}
{\rm d}f(q_{i},Q_{i},t)&=P_{i}{\rm d}Q_{i}-\lambda p_{i}{\rm d}q_{i}+(\lambda\mathcal{H}-\mathcal{H}^{\prime}){\rm d}t.
\end{split}
\end{equation}
The above generating function represents a broader set of transformations which are known as extended canonical transformations. The equations of motions are preserved when systems undergo transformations of such kind.
\section{Phase space}
Let us consider a system, of $N$ particles, which is governed by the Hamiltonian $\mathcal{H}$. At any point of time, the state of the system is defined by the $3N$ positions and $3N$ conjugate momenta. The phase space is a $6N$ dimensional space which consists of all the possible state points which the system can exist in. A volume element (${\rm d}\Gamma$) in the phase space is defined in the following way
\begin{equation}
{\rm d\Gamma}={\rm d}\vec{q}_{1}\cdot\cdot\cdot{\rm d}\vec{q}_{N}{\rm d}\vec{p}_{1}\cdot\cdot\cdot{\rm d}\vec{p}_{N},
\end{equation}
where ${\rm d}\vec{q}={\rm d}q_{x}{\rm d}q_{y}{\rm d}q_{z}$. The phase space trajectory followed by a system in determined by the equations of motion given in Eq.~(\ref{eq:equationsmotion}). An important question which arises is whether the phase space volume remains conserved under extended canonical transformations. This can be understood by calculating the determinant of the corresponding Jacobian matrix, $J$. Let the transformation be of the following form,
\begin{equation}
{\rm d}\vec{q}_{1}\cdot\cdot\cdot{\rm d}\vec{q}_{N}{\rm d}\vec{p}_{1}\cdot\cdot\cdot{\rm d}\vec{p}_{N} \rightarrow {\rm d}\vec{Q}_{1}\cdot\cdot\cdot{\rm d}\vec{Q}_{N}{\rm d}\vec{P}_{1}\cdot\cdot\cdot{\rm d}\vec{P}_{N},
\end{equation} 
where the variables are related through any of the expressions in Eq.~(\ref{eq:generatingfunctions}). The Jacobian $J$ for the above transformation is given by
\begin{equation}
\begin{split}
J&=\frac{\partial \{\vec{Q}_{1}\cdot\cdot\cdot\vec{Q}_{N},\vec{P}_{1}\cdot\cdot\cdot\vec{P}_{N}\}}{\partial \{\vec{q}_{1}\cdot\cdot\cdot\vec{q}_{N},\vec{p}_{1}\cdot\cdot\cdot\vec{p}_{N}\}},\\
&=\frac{\partial \{\vec{Q}_{1}\cdot\cdot\cdot\vec{Q}_{N},\vec{P}_{1}\cdot\cdot\cdot\vec{P}_{N}\}}{\partial \{\vec{Q}_{1}\cdot\cdot\cdot\vec{Q}_{N},\vec{p}_{1}\cdot\cdot\cdot\vec{p}_{N}\}}\frac{\partial \{\vec{Q}_{1}\cdot\cdot\cdot\vec{Q}_{N},\vec{p}_{1}\cdot\cdot\cdot{\rm d}\vec{p}_{N}\}}{\partial \{\vec{q}_{1}\cdot\cdot\cdot\vec{q}_{N},\vec{p}_{1}\cdot\cdot\cdot\vec{p}_{N}\}},\\
&=\frac{\partial \{\vec{P}_{1}\cdot\cdot\cdot\vec{P}_{N}\}}{\partial \{\vec{p}_{1}\cdot\cdot\cdot\vec{p}_{N}\}}\frac{\partial \{\vec{Q}_{1}\cdot\cdot\cdot\vec{Q}_{N}\}}{\partial \{\vec{q}_{1}\cdot\cdot\cdot\vec{q}_{N}\}},\\
&=\frac{\partial \vec{P}_{i}}{\partial \vec{p}_{j}}/\frac{\partial \vec{q}_{i}}{\partial \vec{Q}_{j}},\\
&=\lambda\frac{\partial^{2} F_{3}}{\partial\vec{p}_{j}\partial \vec{Q}_{i}}/\frac{\partial^{2} F_{3}}{\partial\vec{Q}_{j}\partial \vec{p}_{i}}.
\end{split}
\end{equation}
It can be seen that the matrices in the numerator and denominator of the RHS are transposes of each other and have the same determinant value, leading to $|J|=\lambda$. Hence it can be seen that the phase space volume is preserved for canonical transformations ($\lambda=1$). It can be shown that different state points on the trajectory generated through the equations of motion in Eq.~(\ref{eq:equationsmotion}) are canonical transformations of one other. This indicates that the phase space volume is preserved during Hamiltonian dynamics. 

Let the probability of the system existing in a  phase space volume element around a state point  be $f$, then
\begin{equation}
\int^{\infty}_{-\infty}\int^{\infty}_{-\infty}f(\vec{q}_{i},\vec{p}_{i},t) {\rm d}\vec{q}_{i} {\rm d}\vec{p}_{i}=1,
\end{equation}
where $\vec{q}_{i}=\{\vec{q}_{1}\cdot\cdot\cdot\vec{q}_{N}\}, \vec{p}_{i}=\{\vec{p}_{1}\cdot\cdot\cdot\vec{p}_{N}\}$. The above equality assumes that the function $f$ is well behaved and is zero at $-\infty$ and $\infty$. The variation of this distribution function with time can be calculated in the following way,
\begin{equation}
\begin{split}
\frac{{\rm d}}{{\rm d}t}\int\int f(\vec{q}_{i},\vec{p}_{i},t) {\rm d}\vec{q}_{i} {\rm d}\vec{p}_{i}=0.
\end{split}
\end{equation}
A point to remember here is that the phase space volume is independent of time as the dynamical trajectory is a set of canonical transformations. Therefore,
\begin{equation}
\begin{split}
&\int\int \frac{{\rm d}f(\vec{q}_{i},\vec{p}_{i},t) }{{\rm d}t}{\rm d}\vec{q}_{i} {\rm d}\vec{p}_{i}=0,\\
&\int\int \left(\frac{\partial f(\vec{q}_{i},\vec{p}_{i},t) }{\partial t} + \frac{\partial f(\vec{q}_{i},\vec{p}_{i},t) }{\partial \vec{q}_{i}}\vec{\dot{q}}_{i} +\frac{\partial f(\vec{q}_{i},\vec{p}_{i},t) }{\partial \vec{p}_{i}}\vec{\dot{p}}_{i}  \right){\rm d}\vec{q}_{i} {\rm d}\vec{p}_{i}=0,\\
&\int\int \left(\frac{\partial f(\vec{q}_{i},\vec{p}_{i},t) }{\partial t} + \frac{\partial f(\vec{q}_{i},\vec{p}_{i},t) }{\partial \vec{q}_{i}} \frac{\partial \mathcal{H} }{\partial \vec{p}_{i}} -\frac{\partial f(\vec{q}_{i},\vec{p}_{i},t) }{\partial \vec{p}_{i}}\frac{\partial \mathcal{H} }{\partial \vec{q}_{i}}  \right){\rm d}\vec{q}_{i} {\rm d}\vec{p}_{i}=0,\\
&\int\int \left(\frac{\partial f(\vec{q}_{i},\vec{p}_{i},t) }{\partial t} + \frac{\partial f(\vec{q}_{i},\vec{p}_{i},t) }{\partial \vec{q}_{i}} \frac{\partial \mathcal{H} }{\partial \vec{p}_{i}} -\frac{\partial f(\vec{q}_{i},\vec{p}_{i},t) }{\partial \vec{p}_{i}}\frac{\partial \mathcal{H} }{\partial \vec{q}_{i}}  \right){\rm d}\vec{q}_{i} {\rm d}\vec{p}_{i}=0,\\
&\int\int \left(\frac{\partial f(\vec{q}_{i},\vec{p}_{i},t) }{\partial t} + \left\{ f,\mathcal{H}\right\} \right){\rm d}\vec{q}_{i} {\rm d}\vec{p}_{i}=0,\\
&\frac{\partial f(\vec{q}_{i},\vec{p}_{i},t) }{\partial t} + \left\{ f,\mathcal{H}\right\}=0, 
\end{split}
\end{equation}
where $\left\{ f,\mathcal{H}\right\}$ is the Poisson bracket which can be defined in a general way by
\begin{equation}
\left\{ A,B\right\}_{x,y}=\frac{\partial A}{\partial x}\frac{\partial B}{\partial y}-\frac{\partial A}{\partial y}\frac{\partial B}{\partial x}.
\end{equation}
Coming back to the probability distribution,
\begin{equation}
\begin{split}
\frac{\partial f}{\partial t} + \left\{ f,\mathcal{H}\right\}&=0,\\
\frac{\partial f}{\partial t} &= -\left\{ f,\mathcal{H}\right\},\\
\frac{\partial f}{\partial t} &=-i^{2}\left\{ \mathcal{H},f\right\},\\
\frac{\partial f}{\partial t} &=-i\mathscr{L}f,\\
f(t)&=e^{-i\mathscr{L}t}f(0),
\end{split}
\end{equation}
where $\mathscr{L}=i\left\{\mathcal{H},\cdot\cdot\cdot\right\}$ is the Liouville operator. Now let us consider the time dependent variation of a thermodynamic quantity, $B$ which is a function only of the position and conjugate momenta.
\begin{equation}\label{eq:variable}
\begin{split}
\frac{{\rm d}B}{{\rm d}t}&=\left\{B,H\right\},\\
\frac{{\rm d}B}{{\rm d}t}&=i^{2}\left\{H,B\right\},\\
\frac{{\rm d}B}{{\rm d}t}&=i\mathscr{L}B,\\
B(t)&=e^{i\mathscr{L}t}B(0).
\end{split}
\end{equation}
In the above expression, the operator $\mathscr{L}$ can be divided into two parts in the following manner,
\begin{equation}
\begin{split}
i\mathscr{L}&=i\mathscr{L}_{1} + i\mathscr{L}_{2},\\
&=\frac{\partial \mathcal{H} }{\partial \vec{p}_{i}} \frac{\partial }{\partial \vec{q}_{i}} -\frac{\partial \mathcal{H} }{\partial \vec{q}_{i}} \frac{\partial }{\partial \vec{p}_{i}},\\
&=\dot{q}_{i}\frac{\partial }{\partial \vec{q}_{i}} + \dot{p}_{i}\frac{\partial }{\partial \vec{p}_{i}},\\
&=\dot{q}_{i}\frac{\partial }{\partial \vec{q}_{i}} + F_{i}\frac{\partial }{\partial \vec{p}_{i}}.
\end{split}
\end{equation}
Applying the above expression to Eq.~(\ref{eq:variable}), we obtain,
 \begin{equation}\label{eq:variable}
\begin{split}
B(t)&=e^{\left(i\mathscr{L}_{1}t+i\mathscr{L}_{2}t\right)}B(0),\\
&\neq e^{i\mathscr{L}_{1}t}e^{i\mathscr{L}_{2}t}B(0).\\
\end{split}
\end{equation}
The above inequality is due to the fact that the operators $e^{i\mathscr{L}_{1}}$ and $e^{i\mathscr{L}_{2}}$  are non-commutative. The decomposition for the above operator is then performed using the Trotter theorem which gives the following expression
\begin{equation}
\begin{split}
e^{\left(i\mathscr{L}_{1}t+i\mathscr{L}_{2}t\right)}=\left(e^{i\mathscr{L}_{1}\Delta t/2}e^{i\mathscr{L}_{2}\Delta t}e^{i\mathscr{L}_{1}\Delta t/2}\right)^{P} + \mathcal{O}\left((\Delta t)^{3}\right),
\end{split}
\end{equation}
where $\Delta t=t/P$. There are two variations for the above operator (limit of small $\Delta t$),
\begin{equation}\label{eq:trotter}
\begin{split}
e^{\left(i\mathscr{L}_{1}+i\mathscr{L}_{2}\right)\Delta t}=e^{i\mathscr{L}_{1}\Delta t/2}e^{i\mathscr{L}_{2}\Delta t}e^{i\mathscr{L}_{1}\Delta t/2},\\
e^{\left(i\mathscr{L}_{1}+i\mathscr{L}_{2}\right)\Delta t}=e^{i\mathscr{L}_{2}\Delta t/2}e^{i\mathscr{L}_{1}\Delta t}e^{i\mathscr{L}_{2}\Delta t/2}.
\end{split}
\end{equation}
Before moving ahead, we look at the action of the operator $e^{a\partial/\partial x}$ on a function,
\begin{equation}
e^{a\partial/\partial x}f(x)=\sum_{n=0}^{\infty}\frac{a^{n}}{n!}\frac{\partial^{n}}{\partial x^{n}} f(x),
\end{equation}
where the LHS is the Taylor expansion of $f(x+a)$ around $f(x)$, therefore 
\begin{equation}
e^{a\partial/\partial x}f(x)=f(x+a).
\end{equation}
Another point to note from the above expression is that the transformations performed using the above operator are reversible. Let us consider the effect of the two different variations of the operator given in Eq.~(\ref{eq:trotter}) on the system whose state at time $t$ is $\Gamma(\vec{q}_{j},\vec{p}_{j})$. The first variation in Eq.~(\ref{eq:trotter}) leads to the position Verlet algorithm
\begin{equation}
\begin{split}
\Gamma(t+\Delta t)&=e^{i\mathscr{L}_{1}\Delta t/2}e^{i\mathscr{L}_{2}\Delta t}e^{i\mathscr{L}_{1}\Delta t/2}\Gamma(t),\\
\vec{q}_{i}(t+\frac{\Delta t}{2})&=\vec{q}_{i}(t) +\vec{\dot{q}}_{i}(t)\frac{\Delta t}{2},\\
F_{i}&=-\frac{\partial V}{\partial \vec{q}_{i}}\Bigg|_{t+\Delta t/2},\\
\vec{\dot{q}}_{i}(t+\Delta t)&=\vec{\dot{q}}_{i}(t) + \frac{F_{i}}{m}\Delta t,\\
\vec{q}_{i}(t+\Delta t)&=\vec{q}_{i}(t+\frac{\Delta t}{2}) +\vec{\dot{q}}_{i}(t+\Delta t)\frac{\Delta t}{2}.\\
\end{split}
\end{equation}
The second variation leads to the velocity-Verlet algorithm,
\begin{equation}
\begin{split}
\Gamma(t+\Delta t)&=e^{i\mathscr{L}_{2}\Delta t/2}e^{i\mathscr{L}_{1}\Delta t}e^{i\mathscr{L}_{2}\Delta t/2}\Gamma(t),\\
F_{i}&=-\frac{\partial V}{\partial \vec{q}_{i}}\Bigg|_{t},\\
\vec{\dot{q}}_{i}(t+\frac{\Delta t}{2})&=\vec{\dot{q}}_{i}(t) +\frac{F_{i}}{m}\frac{\Delta t}{2},\\
\vec{q}_{i}(t+\Delta t)&=\vec{q}_{i}(t) +\vec{\dot{q}}_{i}\left(t+\frac{\Delta t}{2}\right)\Delta t.\\
\end{split}
\end{equation}
\section{Poisson Brackets}
This section discusses some properties of Poisson brackets
\begin{equation}
\left\{ A,B\right\}_{x,y}=\frac{\partial A}{\partial x}\frac{\partial B}{\partial y}-\frac{\partial A}{\partial y}\frac{\partial B}{\partial x}.
\end{equation}
\begin{equation}
\begin{split}
&\left\{ A,B\right\}_{x,y}=-\left\{ B,A\right\}_{x,y},\\
&\left\{ \lambda A,B\right\}_{x,y}=\lambda\left\{ A,B\right\}_{x,y},\\
&\left\{AC,B\right\}_{x,y}=\frac{\partial AC}{\partial x}\frac{\partial B}{\partial y}-\frac{\partial AC}{\partial y}\frac{\partial B}{\partial x}=\frac{\partial A}{\partial x}C\frac{\partial B}{\partial y}-\frac{\partial A}{\partial y}C\frac{\partial B}{\partial x}+A\frac{\partial C}{\partial x}\frac{\partial B}{\partial y}-A\frac{\partial C}{\partial y}\frac{\partial B}{\partial x}=\left\{A,B\right\}_{x,y}C+A\left\{C,B\right\}_{x,y},\\
&\left\{ A,\left\{B,C\right\}\right\}_{x,y}+\left\{ C,\left\{A,B\right\}\right\}_{x,y}+\left\{ B,\left\{C,A\right\}\right\}_{x,y}=0.
\end{split}
\end{equation}
\end{document}
